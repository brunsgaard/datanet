\documentclass[a4paper]{article}

\usepackage[fancy]{template}
\usepackage{survival-pack}

\setup{%
  subject={Datanet},%
  assignment={Assignment 1},%
  date={29 April, 2013}%
}
\setupLocation[short=DIKU]{Datalogisk institut, Københavns Universitet}
\setupAuthor[addendum={\email{jonas.brunsgaard@gmail.com}}]{Jonas Brunsgaard}

\newcommand{\commandstyle}[0]{{\ttfamily\textbackslash}}
\newcommand{\command}[1]{\texttt{\textbackslash #1}}

\begin{document}

\maketitle
\thispagestyle{first}
\newpage

\section{Theoretical part}
\subsection{Store and Forward}
\subsubsection{Processing and delay}
In addition to propogation delay we have

\begin{description}
    \item[Processing delay] A delay caused by the routers in the network,
        due to the requirement of examining the packet's header information to
        determine where to direct the package. 
    \item[Queue delay] If more packets are going into the same link, the
        router forms a queue, where a packet is hold until it is
        transmitted, this delay is called a queue delay.
    \item[Tranmission delay] The time required to push all of the packet's bits
        into the link. This is often the most most significant delay.
\end{description}

\subsection{Buffers and latency}
\begin{description}
    \item[Part 1] 
    \item[Part 2]
\end{description}

\subsection{HTTP}
\subsubsection{HTTP semantics}

\subsubsection{The case of Deep Packet Inspection}
\begin{description}
    \item[Part 1] '3' could have filtered all requests to \texttt{grooveshark.com} by dropping all packets
        with a destination host being an IP belonging to that domain. This could be done in the network layer without
        breaking the layering model of the network stack.
    \item[Part 2] This is a violation of layering in networks because a layer is inspecting and modifying
        the data passing in a layer above it. To stop this and enforce the layering we may encrypt the HTTP
        stream by using HTTPS (SSL).
\end{description}

\end{document}
