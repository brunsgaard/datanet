Dette dokument udgør opgavesættet for den ordinære eksamen i kurset
``Objektorienteret programmering og design'', blok 1, 2012. Det består af
\pageref{LastPage} nummererede sider. Dokumentet offentliggøres mandag den 14.
januar kl. 9:00 på kursets hjemmeside via KU's kursusadministrationssystem
Absalon.

% \subsection{Afleveringsfrist}

%% The submission deadline is {\bf Friday the 18th of January, 2013, 14:00 CET}.
%% Resubmission is \emph{not} possible, and the deadline \emph{cannot} be
%% extended.

\section{Aflevering}

%% Hand in your solution in digital form, under the folder ``Exam'' on Absalon.
%% You should hand in a single archive file (jar, zip, tar, etc.). The actual
%% structure of the desired archive is explicitly stated at the end of the
%% assignment text. The name of the archive should be your alumni id, e.g.
%% \mono{abc123.zip}.

Besvarelsen skal afleveres senest \textbf{fredag den 18. januar kl. 14:00}. Der
vil \emph{ikke} være mulighed for genaflevering, og afleveringsfristen kan
\emph{ikke} udskydes.

\subsubsection{Procedure}

Besvarelsen afleveres i digital form, under mappen ``Eksamen'' på Absalon. Du
bedes aflevere hele løsningen i en enkelt arkivfil (jar, zip, tar eller
lignende). Arkivfilen skal navngives efter dit KU alumne id, f.eks.
\mono{abc123.zip}. Arkivfilens indhold er beskrevet i sektion
\ref{sec:indhold}.

%% Please make sure that you hand in your \emph{source code} and \emph{not} java
%% bytecode. It is always a good idea to check that all of your files have been
%% uploaded correctly by downloading them again and browsing them
%% yourself.

%% In the unfortunate case that the KU systems are down, send your
%% solution directly to \mono{oleks@diku.dk}. The aforementioned deadline holds
%% regardless.

Skulle KUs systemer være nede i forbindelse med afleveringsfristen,
bedes du sende din løsning direkte til
\mono{oleks@diku.dk}. Afleveringsfristen som angivet ovenfor ændres ikke.

\subsubsection{Indhold}

\label{sec:indhold}

Arkivfilen skal indeholde:

\begin{itemize}

\item En mappe \texttt{./src}, med al din kildetekst.

\item En fil \texttt{./assessment.txt} hvor du udfører en vurdering af din
løsning for hver opgave (mere om dette under selve opgaverne).

\end{itemize}

Det forventes at kildetekstens mappestruktur svarer til den erklærede
pakkestruktur. F.eks. hvis der findes en klasse,
\texttt{spreadsheet.command.Exit}, så forefindes der også en fil
\texttt{./src/spreadsheet/command/Exit.java} med erklæringen \texttt{package
spreadsheet.command} i toppen af filen.

Al kildetekst forventes at være kompatibelt med Java\texttrademark~SE
Development Kit 7, altså version 1.7.0.

Det forventes endvidere at koden generelt er dokumenteret med f.eks. javadoc
kommentarer\footnotemark. Mere specifikt, forventer vi at at alle nye pakker og
klasser er dokumenteret således at en individ uvidende om dette dokument kan
finde rundt i kildekoden. Det er dog ikke strengt taget et krav at du bruger
javadoc til dette. Alle kommentarer og alt kildekode skal være på engelsk.

\footnotetext{javadoc dokumentation:
\url{http://docs.oracle.com/javase/7/docs/technotes/tools/solaris/javadoc.html\#description}.}

Det forventes ikke at der rettes fejl i den udleverede kildekode, men der skal
argumenteres for at fejlen er i den udleverede kildekode og ikke hos jer selv,
hvis det f.eks. forhindrer jeres afprøvning i at virke som forventet.

Inden du afleverer opgaven bør du venligst sikre dig, at din arkivfil
indeholder alle de forventede filer. Det er især vigtigt at tjekke at
arkivfilen indeholder \emph{kildetekst} og altså \emph{ikke} Java bytecode. Det
er desuden altid en god idé at sikre sig at dine filer er blevet lagt op
korrekt, ved at hente arkivfilen fra Absalon igen og kigge indholdet igennem.

\subsubsection{Evaluering}

Det forventes at du er i stand til at evaluere kvaliteten af din løsning. Evaluering skal ske i form af et par korte afsnit for hver opgave i opgaveteksten i den supplerende tekstfil, \texttt{./assessment.txt}.

Evaluering kan f.eks. udføres vha. ``unit testing''. Hvis du benytter dette, forventer vi at du bruger JUnit frameworket til formålet. Der eksister endvidere en skelet
JUnit testfil for hver opgave, nemlig

\begin{enumerate}

\item \texttt{spreadsheet.IfThenElseTest};

\item \texttt{spreadsheet.RangeTest};

\item \texttt{spreadsheet.UndoTest};

\item \texttt{spreadsheet.CopyPastTest}; og

\item \texttt{gui.PlotTest}.

\end{enumerate}

\subsection{Bedømmelse}

Besvarelsen bedømmes efter 7-trinsskalaen ud fra en samlet bedømmelse af,
hvorvidt læringsmålene for kurset er opfyldt (se kursusbeskrivelsen).
Besvarelsen bedømmes indenfor 3 uger efter afleveringsfristen, med ekstern
censur.

\subsection{Selvstændighed i besvarelsen og eksamenssnyd}

Opgavesættet besvares \emph{individuelt}. Det er kun tilladt at stille
opklarende spørgsmål til opgavernes fortolkning, den udleverede kildekode,
eller Java bibliotekkerne til instruktorerne under instruktorvagtordningen
eller på kursets elektroniske diskussionsforum. Spørgsmål på forummet må ikke
indeholde programtekst. Det er \emph{ikke} tilladt for studerende på kurset at
\emph{besvare} spørgsmål på forum --- det er udelukkende instruktorer og
undervisere, der må det. Instruktorerne må endvidere frit redigere indlæg på
forummet hvis de finder dem for specifikke eller på anden vis upassende.

Det er ikke tilladt at diskutere opgavernes fortolkning eller indhold med
andre end de herover nævnte.

Det er ikke tilladt overhovedet at diskutere besvarelse af opgaverne, herunder
afprøvningstilfælde, løsningsmetoder, algoritmer eller konkret programtekst.
Specifikt er følgende ikke tilladt i eksamensperioden, og enhver overtrædelse
vil resultere i indkaldelse til mundtlig overhøring samt overdragelse af sagen
til studielederen til behandling under gældende regler for eksamenssnyd:

\begin{itemize}

\item At vise enhver del af sin besvarelse til andre, herunder specielt
personer, som følger kurset.

\item At vise enhver del af opgavesættet til personer, som ikke er tilknyttet
kurset. Herunder at lægge (dele af) opgaveformuleringer online (fora og
chatrooms inklusive) andetsteds end kursets diskussionsforum.

\item At diskutere opgavesættet eller dets fortolkning med andre personer,
udover de herover nævnte undtagelser.

\item At efterlade opgavesættet eller noter/kladder til ens egen besvarelse
uden opsyn på DIKU. Dette inkluderer at gå fra en ulåst datamat selv i kortere
tid, smide udskrifter i papirkurve på DIKU, eller udskrive opgavesæt og
løsninger på DIKUs printere, da andre har adgang til disse. Vi kan ikke
kontrollere en sådan adfærd, men hvis en anden persons aflevering ligner din,
så vil I begge blive betragtet som ansvarlige for kopieringen.

\item At diskutere med andre eller afskrive fra andre dele eller hele
besvarelser af opgaver fra eksamenssættet.

\item At efterlyse løsninger fra andre.

\item At bruge i øvrigt tilladeligt skriftligt eller mundtligt materiale ud
over kursets undervisningsmateriale uden henvisning til kilden (f.eks.
oplysninger fra Wikipedia, Google Scholar eller lignende).

\end{itemize}

Brugen af \emph{skriftligt} materiale fra offentligt tilgængelige kilder er
\emph{tilladt} under forudsætning af, at kilden var tilgængelig før eksamens
start, og at kilden angives i besvarelsen. For kilder på Internettet skal
komplet URL angives.

Det indskærpes, at alle besvarelser vil blive underlagt både elektronisk og
menneskelig plagiatkontrol. Denne plagiatkontrol er stærk nok til at genkende
ligheder i kode selv om der er lavet forsøg på at skjule denne lighed med
betydningsbevarende omskrivninger i koden.

Hvis eksaminatorer, studieleder og dekan finder det sandsynligt, at disse
regler er overtrådt, så er konsekvensen som minimum, at eksamensresulatet
bliver ændret til -3, og i særligt grove tilfælde kan eksaminanden blive
bortvist fra universitetet.

\subsection{Instruktorvagtordningen}

Der er mulighed for at spørge en instruktor direkte i DIKU's kantine. Der er en
vagt mellem 10 og 13 og mellem 14 og 17 hver dag under eksamen. Fredag er der
kun vagt mellem 10 og 14. Vagtplanen er lagt op ved siden af opgaveteksten på
Absalon.  Bemærk, at der kan forekomme ændringer i vagtplanen.

Instruktoren på vagt vil ellers være nemt identificerbar i kantinen. Forstyr
venligst ikke instruktorene udenfor deres vagter, da de selv har eksamen.

Bemærk også at i visse tilfælde ville jeres medstuderende have gavn af
at få jeres spørgsmål besvaret på diskussionsforummet.

\subsection{Tema}

%% The theme of the exam is to extend thespreadsheet application, ``Regneark''.
%% This is the application we have been developing gradually throughout the course.
%% As an initial code base, we provide a reference implementation, incorporating
%% all of the work done throughout the course. You will find a description of this
%% implementation below.

Formålet med eksamenssættet er at udvide jeres regnearksprogram fra
ugeopgaverne, kaldet ``Regneark''. Som udgangspunkt for løsningen vil der
stilles en færdig implementation af de hidtidige opgaver til rådighed. Det er
tilladt, men ikke anbefalet at bruge ens eget løsning til ugeopgaverne som
grundlag. Den udleverede implementation beskrives i nærmere detaljer nedenfor.

\subsection{Opgaver}

%% The exam consists of five mandatory tasks. We make no statement about the
%% weight of each task towards your final grade.  Indeed, your final grade
%% will depend on your ability to meet the learning objectives\footnotemark, and
%% not necessarily solve all of the exrcises in the exam.

Der vil ikke forekomme en vægtning af de individuelle opgaver, men en
helhedsvurdering af opgaven. Din samlede karakter vil i større grad
afspejle din evne til at møde læringsmålene for kurset\footnotemark,
end om du nødvendigvis kan løse alle opgaverne i eksamenssættet.

\footnotetext{OOPD Kursusbeskrivelse: \url{http://sis.ku.dk/kurser/viskursus.aspx?knr=142243}.}

%% The task descriptions give an outline, but are deliberately incomplete. We
%% expect \emph{you} to fill in the blanks. For instance, we may ask you to add a
%% particular abstract method to a particular abstract base class. You should
%% hereby understand that it is up to you to figure out how to implement the
%% functionality in the concrete subclasses. A short task description does not
%% necessarily mean a simpler task, and vice versa.

Opgavebeskrivelserne vil give dig en kortfattet redegørelse for
opgaven, men er bevidst løst formuleret. Det er altså op til
\emph{dig} at finde ud af hvordan du bedst løser opgaverne. For
eksempel kunne vi bede dig tilføje en bestemt abstrakt metode til en
bestemt abstrakt klasse. Det er hermed op til dig at finde ud af,
hvordan du skal implementere den nødvendige funktionalitet i de
konkrete underklasser. En simpel opgavebeskrivelse er ikke nødvendigvis
det samme som en simpel opgave, og omvendt.

%% In cases where the task description seems absurd or unclear, we expect you to
%% make a simplifying assumption about our intent. You should state your
%% assumption clearly in the documentation of your solution.

I tilfælde hvor opgavebeskrivelsen virker uklar, forventer vi at du
foretager klargørende antagelser om hensigterne med opgaven. I sådanne
tilfælde beder vi dig om at beskrive dine antagelser i dokumentationen
til din løsning vha. javadoc eller andre kommentarer.

%% The exercises are generally independent of one another, with the exception
%of % Exercise \ref{ex:ranges}, which is necessary to successfully implement
%Exercise % \ref{ex:plots}, and is desired, but not strictly speaking
%necessary, to % successfully implement Exercises \ref{ex:undo} and %
%\ref{ex:copy-paste}.

Opgaverne er som udgangspunkt uafhængige af hinanden, med undtagelse af opgave
\ref{ex:ranges}, som er nødvendig for at kunne implementere opgave
\ref{ex:plots} korrekt. En løsning på opgave \ref{ex:ranges} er også
foretrukket, omend ikke streng nødvendig, for at kunne løse opgaverne
\ref{ex:undo} og \ref{ex:copy-paste}.

\subsection{Generelt om oversættelse af engelske begreber}

Inden for programmering (og datalogi generelt) er de fleste fagtermer
og begreber på engelsk. Eftersom eksamensteksten skal være på dansk,
vil der derfor nødvendigvis forekomme en blanding af engelske begreber
oversat til dansk og engelske fagtermer brugt direkte. I tilfælde hvor
engelske begreber er oversat til dansk vil vi så vidt muligt forsøge
at bruge allerede eksisterende danske ord, og i tilfælde hvor
betydningen ikke er helt klar vil vi skrive det engelske ord i
parentes bagefter.

I tilfælde af forvirring omkring brugen af danske udtryk og
oversættelser henviser vi til diskussionforummet eller
instruktorvagten, som vil kunne afklare uklarheder.

\subsection{Terminologi}

\begin{description}[\setleftmargin{60pt}\breaklabel\setlabelstyle{\it}]

%% \item [Evaluation] An evaluation occurs whenever the user calls either the
%% method \texttt{toInt()}, \texttt{toBoolean()}, or \texttt{toString()} on an
%% expression. In other words, an evaluation is a walk of the syntax tree
%% (expression) to produce some sort of a scalar result, be that an integer,
%% boolean or string.

\item [Evaluering] En evaluering forekommer når brugeren kalder en af metoderne
\texttt{toInt()}, \texttt{toBoolean()} og \texttt{toString()} på et udtryk
(Expression). Med andre ord resulterer en evaluering i at syntakstræet gås
igennem og der produceres et resultat af typen \texttt{int}, \texttt{boolean}
eller \texttt{String}.

\item [Uforanderlig] En uforanderlig klasse kan forstås ved det engelske ord
\textit{immutable} og er karakteriseret ved at felter i objekter for klassen
ikke ændrer værdi efter objektet er blevet konstrueret. For eksempel er
\texttt{Position} og \texttt{Expression} uforanderlige, mens
\texttt{Application} ikke er.

\end{description}
