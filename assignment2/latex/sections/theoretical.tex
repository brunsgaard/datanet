\section{Theoretical part}

\subsection{Domain Name System}
\subsubsection{DNS provisions}
The DNS system ensures fault tolerance on a packet level through UDP, which wraps the segments
in a header containing a checksum. On a system level fault tolerance is ensured because the system
is designed as a distributed architecture, making sure that there are no central points of failure.
Both of these techniques also ensure scalability by making making DNS have stateless connections and
spreading requests out over a large amount of servers. Efficiency is then also ensured because UDP has
a very low overhead compared to TCP, and one may query servers that are close.

\subsubsection{DNS lookup and format}
\begin{description}
    % 2-4 sentences
    \item[Part 1:] The CNAME record allows a server to be known through several aliases, such as
        a web-server being named www.domain.tld and a ftp service on the same server being named
        ftp.domain.tld. It may also be used to load-balance requests by returning CNAME records
        for a domain that redirect users to servers that are closer to them.
        %TODO: Can you set all CNAMES in the DNS or do you need something custom?
    % 4-8 sentences
    \item[Part 2:] Iterative lookups work by the root and TLD servers delegating further lookups
        to the requesting party, instead of performing them themselves. This means that the local
        DNS server that a client is asking will ask another server and receive an answer, then acting
        on the information in that answer to fulfill the request (by further lookups). In a recursive
        lookup the server that is being asked is the one that queries further servers for more detailed
        information, and it will only return once it has an authoritative answer.

        Iterative lookups place less strain on the DNS network, because only the local DNS server has
        to maintain state about the request that is being answered. This enables the system to scale
        better. The caches of the local DNS server will also be filled with information on authorative
        and TLD servers, cutting off the root level and speeding up requests.
        %TODO: When and why do we want recursive lookups?
        % Is it something about using it to fill caches?
    % Diagram
    \item[Part 3:]  
\end{description}

\subsection{Transport protocols}
\subsubsection{TCP reliability and utilization}
\begin{description}
    \item[Part 1:] The 3-way handshake ensures that both sides have received the starting segment number of the other side.
        If the final ACK is not sendt, the server cannot be certain that it's starting segment number has actually
        reached the client. If we wish to be sure that the connection is correctly initialized on both sides we must
        therefore use a 3-way-handshake.
    \item[Part 2:] TCP facilitates a full-duplex connection by making the ACK a field in the header of segments, thus
        allowing mixing an answer to a segment with a new data segment to be delivered to the other side. The setup also
        initializes the link in both directions, so both sides are ready to send to each other.
\end{description}

\subsubsection{Reliability vs overhead}
\begin{description}
    \item[Part 1:] TCP adds overhead both in the header (being approx. 20 bytes in size, compared to 8 bytes for UDP), and
        by requiring that all segments are acknowledged and optionally re-sent. It must also negotiate segment numbers before
        starting to transmit data, while UDP starts blasting data immediately.
    \item[Part 2:] 
\end{description}

\subsubsection{Use of transport protocols}
\begin{description}
    \item[DNS part 1:] DNS employs the UDP transport protocol. This is because it has a low overhead, does not require
        the connections to negotiate before sending data, or keep state on a connection. The result of this is that the DNS
        protocol scales much better than if it was sent over a stateful transport protocol like TCP. %TODO: More?
    % 3-8 sentences, including scenario
    \item[DNS part 2:] 

    % TODO: May data be mixed with the FIN packets?
    \item[HTTP part 1:] The minimum number of packets is 3 for the handshake (where the last one can hold the request data because
        of the large MTU), 2 for the server ACK with corresponding data and the client ACK, and 4 for the closing of the connection.
        This results in a minimal number of 9 packets from a TCP perspective.
    % TODO: Correct if packages change
    \item[HTTP part 2:] Only 1/9 of the packets used for the connection actually transmit data. The justification for using
        TCP despite of this overhead is the fact that it provides the reliability needed by HTTP (as discussed in the next part), and
        that it allows for keeping connections open and thus reducing the setup overhead.
    \item[HTTP part 3:] The choice of UDP as a transport protocol for HTTP would be poor, because the guarantees provided by
        TCP are needed for HTTP transmissions. The use of HTTP does not allow for packets to be completely dropped (HTTP is not
        loss-tolerant), and the data stream must be presented in-order for the documents to make sense. In addition to this the
        HTTP protocol is naturally structured such that it fits the TCP model well, by defining a stream of requests and responses.
        %TODO: More?
\end{description}

\subsection{TCP: Principles and practice}
% Specify the point of view, is it from the client's or server's side (or both)?
\subsubsection{TCP headers}
\begin{description}
    \item[Part 1.1:] 
    \item[Part 1.2:] 
    \item[Part 1.3:] 
    \item[Part 1.4:] 
    \item[Part 2:] 
    \item[Part 3:] 
\end{description}

\subsubsection{High performance TCP}
\begin{description}
    \item[Part 1:] % Answer with a number and a sentence
    \item[Part 2:] % Account for the calculations
    \item[Part 3:] 
\end{description}

\subsubsection{Flow and Congestion control}
\begin{description}
    \item[Part 1:] 
    \item[Part 2:] 
    \item[Part 3:] % At most 10 sentences
    \item[Part 4:] 
\end{description}
