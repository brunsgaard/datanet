\section{Theoretical part}

\subsection{Domain Name System}
\subsubsection{DNS provisions} % 2-4 sentences, too much?
The DNS system ensures fault tolerance on a packet level through UDP, which wraps the segments
in a header containing a checksum. On a system level fault tolerance is ensured because the system
is designed as a distributed architecture, making sure that there are no central points of failure.
Both of these techniques also ensure scalability by making making DNS have stateless connections and
spreading requests out over a large amount of servers. Efficiency is then also ensured because UDP has
a very low overhead compared to TCP, and one may query servers that are close.

\subsubsection{DNS lookup and format}
\begin{description}
    % 2-4 sentences
    \item[Part 1:] The CNAME record allows a server to be known through several aliases, such as
        a web-server being named www.domain.tld and a ftp service on the same server being named
        ftp.domain.tld. It may also be used to load-balance requests by returning CNAME records
        for a domain that redirect users to servers that are closer to them.
        %TODO: Can you set all CNAMES in the DNS or do you need something custom?
    % 4-8 sentences
    \item[Part 2:] Iterative lookups work by the root and TLD servers delegating further lookups
        to the requesting party, instead of performing them themselves. This means that the local
        DNS server that a client is asking will ask another server and receive an answer, then acting
        on the information in that answer to fulfill the request (by further lookups). In a recursive
        lookup the server that is being asked is the one that queries further servers for more detailed
        information, and it will only return once it has an authoritative answer.

        Iterative lookups place less strain on the DNS network, because only the local DNS server has
        to maintain state about the request that is being answered. This enables the system to scale
        better. The caches of the local DNS server will also be filled with information on authorative
        and TLD servers, cutting off the root level and speeding up requests.
        %TODO: When and why do we want recursive lookups?
        % Is it something about using it to fill caches?
    % diagram
    \item[Part 3:]  
\end{description}

\subsection{Transport protocols}
\subsubsection{TCP reliability and utilization}
\begin{description}
    \item[Part 1:] The 3-way handshake ensures that both sides have received the starting segment number of the other side.
        If the final ACK is not sendt, the server cannot be certain that it's starting segment number has actually
        reached the client. If we wish to be sure that the connection is correctly initialized on both sides we must
        therefore use a 3-way-handshake.
        % We can do the "simultaneous-open", but that is crazy
    \item[Part 2:] TCP facilitates a full-duplex connection by making the ACK a field in the header of segments, thus
        allowing mixing an answer to a segment with a new data segment to be delivered to the other side. The setup also
        initializes the link in both directions, so both sides are ready to send to each other.
\end{description}

\subsubsection{Reliability vs overhead}
\begin{description}
    \item[Part 1:] TCP adds overhead both in the header (being approx. 20 bytes in size, compared to 8 bytes for UDP), and
        by requiring that all segments are acknowledged and optionally re-sent. It must also negotiate segment numbers before
        starting to transmit data, while UDP starts blasting data immediately.
    \item[Part 2:] 
\end{description}

\subsubsection{Use of transport protocols}

\subsection{TCP: Principles and practice}
\subsubsection{TCP headers}
\subsubsection{High performance TCP}
\subsubsection{Flow and Congestion control}
