\section{Practical part}

\subsection{Notes}
\begin{description}
    \item[large response on USERLIST]
        If there are many users, the response from the server to for a USERLIST request, could be larger than the current BUFFERSIZE (1024 bytes). In the python implementation there is no way to known if the whole message was received in the first call to \texttt{recv}, if the string obtained has length of \texttt{BUFFERSIZE}. Therefore after the client has received the response on the USERLIST request, it disables blocking receiving, and reads on the socket until no more can be read. This issue could be avoided by using and end-marker as discussed in assignment 1.
    \item[Max nick length]
        We added a limitation on the maximum length of a nick to 32 characters. Allowing users to have arbitrarily long nicks did not suit us very well.
    \item[Invalid nick]
        An error code was added to the protocol for handling invalid nick names. This includes two cases, when a user tries to include a comma in his/her username, or if the length exceeds the maximum length. This got the message ``103 INVALID NICK''.
    \item[Disconnection on handshake fail]
        If the handshake does not succeed, the client is disconnected -- either with a ``101 TAKEN'', ``102 HANDSHAKE EXPECTED'' or ``103 INVALID NICK'' reply. This is a minor change to the proposed protocol, where the client only should be disconnected when the nick was taken.
    \item[Protocol fail]
        We would also like to point our that the protocol is a bit redundant, as the USERLIST command lists the IP of each peer, which might not be needed, as this information can be obtained with the LOOKUP command. Instead the USERLIST command should just have return a list of users.
\end{description}

\subsection{Questions}
\begin{description} %
    \item[Question 1]
        It would be quite easy to reverse engineer the protocol by packet sniffing -- as it isn't a very complicated protocol, and every request and response even have nice human understandable messages. It would be a bit harder if the server only responded with status codes, but it would only make it slightly harder.
        One could use encryption to make it much harder to reverse engineer the protocol by sniffing packets. This could be accomplished by the use of TLS/SSL.

    \item[Question 2]
        Our implementation is not bullet proof, but will not fall apart by any unexpected event. One would be able to perform a DoS attack by connecting so many clients to the server that it can't create any new sockets for incoming users. The clients would have to perform the HELLO handshake to initialize the connection, as SYN floods can be avoided by implementing SYN cookies (RFC 4987). If the service used end-markers, one could take counter measures by putting an upper limit on the length of a single request, as the only commands that take more than a single token also takes a nick, which is limited to 32 characters, one would know by certainty that a request over 39 (7+32)characters is invalid. Thus, one does not need to fill many buffers with randomly generated characters, but can immediately respond with ``500 BAD FORMAT''. If a user repeatedly exhibits this malicious behavior, one could add a rule to a firewall blocking that IP for a certain amount of time.

    \item[Question 3]
        There is currently no way for the name server to keep track of the active peers. A user may make an initial handshake, get the userlist, make a lookup and then go on chatting for hours on end without ever sending a message to the name server again. In this case the user would appear to be idle to the name server -- which in reality is not true -- so the name server should definitely not ``kick'' idle peers.

        If a user closes the client program, or even the computer it is running on, the socket connection to the name server will be closed, and the client removed from the name server.  One could also implement an ``away status'' as seen in Skype, sending a message to the server if the user in inactive for more than 5 minutes, and informing the server when the client returns.

    \item[Question 4]
        Generally it seems to be a good idea to have a central name server. As peer only use the server for seeing who's online, and looking up their IP, most of the workload (sending messages) is fully peer-to-peer. It is also easier to push updates to the server if a bug was encountered. If one did not want to use a central name server, one could use purely peer-to-peer with the Chord algorithm for distributed hash tables. This does prove a problem in coming up with a smart scheme when a new client wants to connect to the network, and making the process of managing who's online much much harder - race conditions could easily arise where the same nick was given to two different people by different ``servers'' at the same time. Having a centralized name server does pose a disadvantage as is represents a single point of failure. If the server is down, no one can chat (or at least not establish new connections). This makes it easier to do a DoS attack on the service. A central name server would not be able to handle a million users by it self -- a distributed service might.
\end{description}
