\section{Theoretical part}

\subsection{The Internet Protocol}
\subsubsection{Addresses and network masks}
\begin{description}
    \item[Part 1.1:] From a router's point of view the purpose of network masks is to
        reduce the size of the forwarding table. Because all IP addresses in the same network
        have the same prefix, a router only needs to store the prefix.
    \item[Part 1.2:] Network masks may be expressed in the same dotted quad notation
        that IP addresses are expressed in (because they are also 4 bytes long).
        255.225.255.0 is not a valid network mask, because it is not a continuous
        prefix of 1-bits. In the slash-notation we would interpret /28 as a prefix
        of 28 1-bits, or the network mask 255.255.255.240.
    \item[Part 1.3:] The difference is that the network prefix was constrained to be of length
        8, 16 or 24 bits in classful addressing, and can be of any length in classless
        interdomain routing.
    \item[Part 1.4:] The network address is simply the IP anded with the network mask (byte form),
        and the broadcast address is the network address plus \\$2^{32-\text{prefixlength}} - 1$.
        The size of the network is $2^{\text{prefixlength}} - 2$ (subtracting 2 for the network
        and broadcast address). The available addresses are found in the interval between the network
        and broadcast address.

    %TODO: When I get a nice tool for it, very soon
    \item[Part 2.1:]
        Size of the network: 30\\
        Network address: 130.225.165.0\\
        Network mask: 255.255.255.224\\
        Broadcast address: 130.225.156.31\\
        First, fifth, last address: 130.225.165.1, 130.225.165.5, 130.225.165.30
    \item[Part 2.2:]
        Size of the network: 510\\
        Network address: 10.0.42.0\\
        Network mask: 255.255.254.0\\
        Broadcast address: 10.0.43.255\\
        First, fifth, last address: 10.0.42.1, 10.0.42.5, 10.0.43.254
    \item[Part 2.3:]
        Size of the network: 1\\
        Network address:?\\
        Network mask: 255.255.255.255\\
        Broadcast address:?\\
        First, fifth, last address:?
    \item[Part 2.4:]
        Size of the network: 16382\\
        Network address: \\
        Network mask: 255.255.192.0\\
        Broadcast address: \\
        First, fifth, last address: 
\end{description}

\subsubsection{Network Address Translation}
\begin{description}
    \item[Part 1:] NAT-enabled routers handle multiple connections by maintaining a NAT translation
        table that maps pairs of internal IPs and ports to external ports. Each connection is given
        an external port in the router, such that traffic going back to the router can be de-multiplexed
        and sent to the appropriate internal IP and port.
    \item[Part 2:] Because ports are limited to 16 bits, there are only 65536 possible ports to choose from.
        If 65536 long lived connections are opened by the internal hosts, no further mappings can be made in
        the NAT translation table. The routers ports are exhausted, and no further connections can be made.
    \item[Part 3:] NAT breaks the layering principle of the Internet because it alters the contents of the 
        transport layer port field. This is not allowed in the layering model, because a router only handles
        the layers up to the network layer.
\end{description}

\subsection{Distributed Hash Tables}
\begin{description}
    \item[Part 1:] 
    \item[Part 2:] 
    \item[Part 3:] 
\end{description}

\subsection{SSH Tunneling}
\subsubsection{{\sc Echoserver} $\leftrightarrow$ {\sc Client}}
\begin{description}
    \item[Question 1:] 
    \item[Question 2:] 
    \item[Question 3:] 
    \item[Question 4:] 
    \item[Question 5:] 
    \item[Question 6:] 
    \item[Question 7:] 
    \item[Question 8:] 
\end{description}
